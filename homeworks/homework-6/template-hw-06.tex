% Options for packages loaded elsewhere
\PassOptionsToPackage{unicode}{hyperref}
\PassOptionsToPackage{hyphens}{url}
%
\documentclass[
]{article}
\usepackage{lmodern}
\usepackage{amssymb,amsmath}
\usepackage{ifxetex,ifluatex}
\ifnum 0\ifxetex 1\fi\ifluatex 1\fi=0 % if pdftex
  \usepackage[T1]{fontenc}
  \usepackage[utf8]{inputenc}
  \usepackage{textcomp} % provide euro and other symbols
\else % if luatex or xetex
  \usepackage{unicode-math}
  \defaultfontfeatures{Scale=MatchLowercase}
  \defaultfontfeatures[\rmfamily]{Ligatures=TeX,Scale=1}
\fi
% Use upquote if available, for straight quotes in verbatim environments
\IfFileExists{upquote.sty}{\usepackage{upquote}}{}
\IfFileExists{microtype.sty}{% use microtype if available
  \usepackage[]{microtype}
  \UseMicrotypeSet[protrusion]{basicmath} % disable protrusion for tt fonts
}{}
\makeatletter
\@ifundefined{KOMAClassName}{% if non-KOMA class
  \IfFileExists{parskip.sty}{%
    \usepackage{parskip}
  }{% else
    \setlength{\parindent}{0pt}
    \setlength{\parskip}{6pt plus 2pt minus 1pt}}
}{% if KOMA class
  \KOMAoptions{parskip=half}}
\makeatother
\usepackage{xcolor}
\IfFileExists{xurl.sty}{\usepackage{xurl}}{} % add URL line breaks if available
\IfFileExists{bookmark.sty}{\usepackage{bookmark}}{\usepackage{hyperref}}
\hypersetup{
  pdftitle={Homework 6 Template, STA 360/602},
  pdfauthor={Rebecca C. Steorts},
  hidelinks,
  pdfcreator={LaTeX via pandoc}}
\urlstyle{same} % disable monospaced font for URLs
\usepackage[margin=1in]{geometry}
\usepackage{color}
\usepackage{fancyvrb}
\newcommand{\VerbBar}{|}
\newcommand{\VERB}{\Verb[commandchars=\\\{\}]}
\DefineVerbatimEnvironment{Highlighting}{Verbatim}{commandchars=\\\{\}}
% Add ',fontsize=\small' for more characters per line
\usepackage{framed}
\definecolor{shadecolor}{RGB}{248,248,248}
\newenvironment{Shaded}{\begin{snugshade}}{\end{snugshade}}
\newcommand{\AlertTok}[1]{\textcolor[rgb]{0.94,0.16,0.16}{#1}}
\newcommand{\AnnotationTok}[1]{\textcolor[rgb]{0.56,0.35,0.01}{\textbf{\textit{#1}}}}
\newcommand{\AttributeTok}[1]{\textcolor[rgb]{0.77,0.63,0.00}{#1}}
\newcommand{\BaseNTok}[1]{\textcolor[rgb]{0.00,0.00,0.81}{#1}}
\newcommand{\BuiltInTok}[1]{#1}
\newcommand{\CharTok}[1]{\textcolor[rgb]{0.31,0.60,0.02}{#1}}
\newcommand{\CommentTok}[1]{\textcolor[rgb]{0.56,0.35,0.01}{\textit{#1}}}
\newcommand{\CommentVarTok}[1]{\textcolor[rgb]{0.56,0.35,0.01}{\textbf{\textit{#1}}}}
\newcommand{\ConstantTok}[1]{\textcolor[rgb]{0.00,0.00,0.00}{#1}}
\newcommand{\ControlFlowTok}[1]{\textcolor[rgb]{0.13,0.29,0.53}{\textbf{#1}}}
\newcommand{\DataTypeTok}[1]{\textcolor[rgb]{0.13,0.29,0.53}{#1}}
\newcommand{\DecValTok}[1]{\textcolor[rgb]{0.00,0.00,0.81}{#1}}
\newcommand{\DocumentationTok}[1]{\textcolor[rgb]{0.56,0.35,0.01}{\textbf{\textit{#1}}}}
\newcommand{\ErrorTok}[1]{\textcolor[rgb]{0.64,0.00,0.00}{\textbf{#1}}}
\newcommand{\ExtensionTok}[1]{#1}
\newcommand{\FloatTok}[1]{\textcolor[rgb]{0.00,0.00,0.81}{#1}}
\newcommand{\FunctionTok}[1]{\textcolor[rgb]{0.00,0.00,0.00}{#1}}
\newcommand{\ImportTok}[1]{#1}
\newcommand{\InformationTok}[1]{\textcolor[rgb]{0.56,0.35,0.01}{\textbf{\textit{#1}}}}
\newcommand{\KeywordTok}[1]{\textcolor[rgb]{0.13,0.29,0.53}{\textbf{#1}}}
\newcommand{\NormalTok}[1]{#1}
\newcommand{\OperatorTok}[1]{\textcolor[rgb]{0.81,0.36,0.00}{\textbf{#1}}}
\newcommand{\OtherTok}[1]{\textcolor[rgb]{0.56,0.35,0.01}{#1}}
\newcommand{\PreprocessorTok}[1]{\textcolor[rgb]{0.56,0.35,0.01}{\textit{#1}}}
\newcommand{\RegionMarkerTok}[1]{#1}
\newcommand{\SpecialCharTok}[1]{\textcolor[rgb]{0.00,0.00,0.00}{#1}}
\newcommand{\SpecialStringTok}[1]{\textcolor[rgb]{0.31,0.60,0.02}{#1}}
\newcommand{\StringTok}[1]{\textcolor[rgb]{0.31,0.60,0.02}{#1}}
\newcommand{\VariableTok}[1]{\textcolor[rgb]{0.00,0.00,0.00}{#1}}
\newcommand{\VerbatimStringTok}[1]{\textcolor[rgb]{0.31,0.60,0.02}{#1}}
\newcommand{\WarningTok}[1]{\textcolor[rgb]{0.56,0.35,0.01}{\textbf{\textit{#1}}}}
\usepackage{graphicx}
\makeatletter
\def\maxwidth{\ifdim\Gin@nat@width>\linewidth\linewidth\else\Gin@nat@width\fi}
\def\maxheight{\ifdim\Gin@nat@height>\textheight\textheight\else\Gin@nat@height\fi}
\makeatother
% Scale images if necessary, so that they will not overflow the page
% margins by default, and it is still possible to overwrite the defaults
% using explicit options in \includegraphics[width, height, ...]{}
\setkeys{Gin}{width=\maxwidth,height=\maxheight,keepaspectratio}
% Set default figure placement to htbp
\makeatletter
\def\fps@figure{htbp}
\makeatother
\setlength{\emergencystretch}{3em} % prevent overfull lines
\providecommand{\tightlist}{%
  \setlength{\itemsep}{0pt}\setlength{\parskip}{0pt}}
\setcounter{secnumdepth}{-\maxdimen} % remove section numbering
% Custom definitions
% To use this customization file, insert the line "\input{custom}" in the header of the tex file.

% Formatting




% Packages

 \usepackage{amssymb,latexsym}
\usepackage{amssymb,amsfonts,amsmath,latexsym,amsthm, bm}
%\usepackage[usenames,dvipsnames]{color}
%\usepackage[]{graphicx}
%\usepackage[space]{grffile}
\usepackage{mathrsfs}   % fancy math font
% \usepackage[font=small,skip=0pt]{caption}
%\usepackage[skip=0pt]{caption}
%\usepackage{subcaption}
%\usepackage{verbatim}
%\usepackage{url}
%\usepackage{bm}
\usepackage{dsfont}
\usepackage{multirow}
%\usepackage{extarrows}
%\usepackage{multirow}
%% \usepackage{wrapfig}
%% \usepackage{epstopdf}
%\usepackage{rotating}
%\usepackage{tikz}
%\usetikzlibrary{fit}					% fitting shapes to coordinates
%\usetikzlibrary{backgrounds}	% drawing the background after the foreground


% \usepackage[dvipdfm,colorlinks,citecolor=blue,linkcolor=blue,urlcolor=blue]{hyperref}
%\usepackage[colorlinks,citecolor=blue,linkcolor=blue,urlcolor=blue]{hyperref}
%%\usepackage{hyperref}
%\usepackage[authoryear,round]{natbib}


%  Theorems, etc.

%\theoremstyle{plain}
%\newtheorem{theorem}{Theorem}[section]
%\newtheorem{corollary}[theorem]{Corollary}
%\newtheorem{lemma}[theorem]{Lemma}
%\newtheorem{proposition}[theorem]{Proposition}
%\newtheorem{condition}[theorem]{Condition}
% \newtheorem{conditions}[theorem]{Conditions}

%\theoremstyle{definition}
%\newtheorem{definition}[theorem]{Definition}
%% \newtheorem*{unnumbered-definition}{Definition}
%\newtheorem{example}[theorem]{Example}
%\theoremstyle{remark}
%\newtheorem*{remark}{Remark}
%\numberwithin{equation}{section}




% Document-specific shortcuts
\newcommand{\btheta}{{\bm\theta}}
\newcommand{\bbtheta}{{\pmb{\bm\theta}}}

\newcommand{\commentary}[1]{\ifx\showcommentary\undefined\else \emph{#1}\fi}

\newcommand{\term}[1]{\textit{\textbf{#1}}}

% Math shortcuts

% Probability distributions
\DeclareMathOperator*{\Exp}{Exp}
\DeclareMathOperator*{\TExp}{TExp}
\DeclareMathOperator*{\Bernoulli}{Bernoulli}
\DeclareMathOperator*{\Beta}{Beta}
\DeclareMathOperator*{\Ga}{Gamma}
\DeclareMathOperator*{\TGamma}{TGamma}
\DeclareMathOperator*{\Poisson}{Poisson}
\DeclareMathOperator*{\Binomial}{Binomial}
\DeclareMathOperator*{\NormalGamma}{NormalGamma}
\DeclareMathOperator*{\InvGamma}{InvGamma}
\DeclareMathOperator*{\Cauchy}{Cauchy}
\DeclareMathOperator*{\Uniform}{Uniform}
\DeclareMathOperator*{\Gumbel}{Gumbel}
\DeclareMathOperator*{\Pareto}{Pareto}
\DeclareMathOperator*{\Mono}{Mono}
\DeclareMathOperator*{\Geometric}{Geometric}
\DeclareMathOperator*{\Wishart}{Wishart}

% Math operators
\DeclareMathOperator*{\argmin}{arg\,min}
\DeclareMathOperator*{\argmax}{arg\,max}
\DeclareMathOperator*{\Cov}{Cov}
\DeclareMathOperator*{\diag}{diag}
\DeclareMathOperator*{\median}{median}
\DeclareMathOperator*{\Vol}{Vol}

% Math characters
\newcommand{\R}{\mathbb{R}}
\newcommand{\Z}{\mathbb{Z}}
\newcommand{\E}{\mathbb{E}}
\renewcommand{\Pr}{\mathbb{P}}
\newcommand{\I}{\mathds{1}}
\newcommand{\V}{\mathbb{V}}

\newcommand{\A}{\mathcal{A}}
%\newcommand{\C}{\mathcal{C}}
\newcommand{\D}{\mathcal{D}}
\newcommand{\Hcal}{\mathcal{H}}
\newcommand{\M}{\mathcal{M}}
\newcommand{\N}{\mathcal{N}}
\newcommand{\X}{\mathcal{X}}
\newcommand{\Zcal}{\mathcal{Z}}
\renewcommand{\P}{\mathcal{P}}

\newcommand{\T}{\mathtt{T}}
\renewcommand{\emptyset}{\varnothing}


% Miscellaneous commands
\newcommand{\iid}{\stackrel{\mathrm{iid}}{\sim}}
\newcommand{\matrixsmall}[1]{\bigl(\begin{smallmatrix}#1\end{smallmatrix} \bigr)}

\newcommand{\items}[1]{\begin{itemize} #1 \end{itemize}}

\newcommand{\todo}[1]{\emph{\textcolor{red}{(#1)}}}

\newcommand{\branch}[4]{
\left\{
	\begin{array}{ll}
		#1  & \mbox{if } #2 \\
		#3 & \mbox{if } #4
	\end{array}
\right.
}

% approximately proportional to
\def\app#1#2{%
  \mathrel{%
    \setbox0=\hbox{$#1\sim$}%
    \setbox2=\hbox{%
      \rlap{\hbox{$#1\propto$}}%
      \lower1.3\ht0\box0%
    }%
    \raise0.25\ht2\box2%
  }%
}
\def\approxprop{\mathpalette\app\relax}

% \newcommand{\approptoinn}[2]{\mathrel{\vcenter{
  % \offinterlineskip\halign{\hfil$##$\cr
    % #1\propto\cr\noalign{\kern2pt}#1\sim\cr\noalign{\kern-2pt}}}}}

% \newcommand{\approxpropto}{\mathpalette\approptoinn\relax}





\ifluatex
  \usepackage{selnolig}  % disable illegal ligatures
\fi

\title{Homework 6 Template, STA 360/602}
\author{Rebecca C. Steorts}
\date{}

\begin{document}
\maketitle

\begin{enumerate}
\def\labelenumi{\arabic{enumi}.}
\setcounter{enumi}{1}
\tightlist
\item
  Researchers are studying the length of life (lifetime) following a
  particular medical intervention, such as a new surgical treatment for
  heart disease, where the study consists of 12 patients. Specifically,
  the number of years before death for each is
  \[3.4, 2.9, 1.2+, 1.4, 3.2, 1.8, 4.6, 1.7+, 2.0+, 1.4+, 2.8, 0.6+\]
  where the \(+\) indicates that the patient was alive after \(x\)
  years, but the researchers lost contact with the patient after that
  point in time.
\end{enumerate}

One way we can model this data is in the following way:

\begin{align}
  &X_i = \branch{Z_i}{Z_i \leq c_i}{c_i}{Z_i > c_i}\\
     & Z_1,\ldots,Z_n|\theta\,\stackrel{iid}{\sim} \,\Ga(r,\theta)\\
    & \theta\sim \Ga(a, b)
\end{align} where \(a\), \(b\), and \(r\) are known. In addition, we
know:

\begin{itemize}
\item $c_i$ is the censoring time for patient $i$, which is fixed, but known only if censoring occurs.
\item $X_i$ is the observation
\begin{itemize}
\item if the lifetime is less than $c_i$ then we get to observe it ($X_i = Z_i$),
\item otherwise
        all we know is the lifetime is greater than $c_i$ ($X_i = c_i$).
        \end{itemize}
    \item $\theta$ is the parameter of interest---the rate parameter for the lifetime distribution.
    \item $Z_i$ is the lifetime for patient $i$, however, this is not directly observed.
    \end{itemize}

The probability density function (pdf) associated consists of two point
masses:one at \(Z_i\) and one at \(c_i\). The formula is \begin{align*}
p(x_i|z_i) = \bm{1}(x_i = z_i)\bm{1}(z_i \leq c_i) + \bm{1}(x_i=c_i)\bm{1}(z_i > c_i).
\end{align*}.

Now we can easily find the full conditionals (derived in class and
reproduced below). Notice that \(z_i\) is conditionally independent of
\(z_j\) given \(\theta\) for \(i \neq j\). This implies that \(x_i\) is
conditionally independent of \(x_j\) given \(z_i\) for \(i \neq j\). Now
we have \begin{align*}
p(z_i|z_{-i},x_{1:n},\theta) &= p(z_i|x_i,\theta) \\
&\underset{z_i}{\propto} p(z_i,x_i,\theta) \\
&= p(\theta)p(z_i|\theta)p(x_i|z_i,\theta) \\
&\underset{z_i}{\propto} p(z_i|\theta)p(x_i|z_i,\theta) \\
&= p(z_i|\theta)p(x_i|z_i).
\end{align*}

There are now two cases to consider. If \(x_i \neq c_i\), then
\(p(z_i|\theta)p(x_i|z_i)\) is only non-zero when \(z_i = x_i\). The
density devolves to a point mass at \(x_i\). This corresponds to the
case where \(z_i\) is observed, so \(x_i\) is the observed value and we
should always sample this value. Practically speaking, we do not sample
this value when running the Gibbs sampler.

If \(x_i = c_i\), then the density becomes
\(p(x_i|z_i) = \bm{1}(z_i > c_i)\), so \begin{align*}
p(z_i|\hdots) \propto p(z_i|\theta)\bm{1}(z_i>c_i),
\end{align*} which is a truncated Gamma.

For the Gibbs sampler, we will use the current value of \(\theta\) to
impute the censored data. We will sample from the truncated gamma using
a modified version of the iverse CDF trick. For the censored values of
\(Z_i\) we know \(c_i\). If we know \(\theta\) (which we will in a
Gibbs' sampler), we know the distribution of
\(Z_i|\theta \sim Gamma(r,\theta)\). Let \(F\) be the CDF of this
distribution. Suppose we truncate this distribution to \((c,\infty)\).
The new CDF is \begin{align*}
P(Z_i < z) &= \frac{F(z) - F(c)}{1 - F(c)}.
\end{align*} Therefore \(Y\) is a sample from the truncated Gamma, as
desired.

In the actual code for the Gibbs' sampler we do not sample the observed
values. We simply impute the censored values using the method above.

You will find code below (that is also taken from class ) that will help
you with the remainder of the problem.

\begin{enumerate}
\def\labelenumi{\arabic{enumi}.}
\item
  (5 points) Write code to produce trace plots and running average plots
  for the censored values for 40 iterations. Do these diagnostic plots
  suggest that you have run the sampler long enough? Explain.
\item
  (5 points) Now run the chain for 10,000 iterations and update your
  diagnostic plots (traceplots and running average plots). Report your
  findings for both traceplots and the running average plots for
  \(\theta\) and the censored values. Do these diagnostic plots suggest
  that you have run the sampler long enough? Explain.
\item
  (5 points) Give plots of the estimated density of
  \(\theta \mid \cdots\) and \(z_9 \mid \cdots\). Be sure to give brief
  explanations of your results and findings. (Present plots for 10,000
  iterations).
\item
  (5 points) Finally, let's suppose that \(r=10,a=1,b=100.\) Do the
  posterior densities in part (c) change for \(\theta \mid \cdots\) and
  \(z_9 \mid \cdots?\) Do the associated posterior densities change when
  \(r=10, a=100,b=1?\) Please provide plots and an explanation to back
  up your answer. (Use 10,000 iterations for the Gibbs sampler).
\end{enumerate}

\begin{Shaded}
\begin{Highlighting}[]
\NormalTok{knitr}\OperatorTok{::}\NormalTok{opts\_chunk}\OperatorTok{$}\KeywordTok{set}\NormalTok{(}\DataTypeTok{cache=}\OtherTok{TRUE}\NormalTok{)}
\KeywordTok{library}\NormalTok{(xtable)}
\CommentTok{\# Samples from a truncated gamma with}
\CommentTok{\# truncation (t, infty), shape a, and rate b}
\CommentTok{\# Input: t,a,b}
\CommentTok{\# Output: truncated Gamma(a,b)}
\NormalTok{sampleTrunGamma \textless{}{-}}\StringTok{ }\ControlFlowTok{function}\NormalTok{(t, a, b)\{}
  \CommentTok{\# This function samples from a truncated gamma with}
  \CommentTok{\# truncation (t, infty), shape a, and rate b}
\NormalTok{  p0 \textless{}{-}}\StringTok{ }\KeywordTok{pgamma}\NormalTok{(t, }\DataTypeTok{shape =}\NormalTok{ a, }\DataTypeTok{rate =}\NormalTok{ b)}
\NormalTok{  x \textless{}{-}}\StringTok{ }\KeywordTok{runif}\NormalTok{(}\DecValTok{1}\NormalTok{, }\DataTypeTok{min =}\NormalTok{ p0, }\DataTypeTok{max =} \DecValTok{1}\NormalTok{)}
\NormalTok{  y \textless{}{-}}\StringTok{ }\KeywordTok{qgamma}\NormalTok{(x, }\DataTypeTok{shape =}\NormalTok{ a, }\DataTypeTok{rate =}\NormalTok{ b)}
  \KeywordTok{return}\NormalTok{(y)}
\NormalTok{\}}
\CommentTok{\# Gibbs sampler for censored data}
\CommentTok{\# Inputs:}
  \CommentTok{\# this function is a Gibbs sampler}
  \CommentTok{\# z is the fully observe data}
  \CommentTok{\# c is censored data}
  \CommentTok{\# n.iter is number of iterations}
  \CommentTok{\# init.theta and init.miss are initial values for sampler}
  \CommentTok{\# r,a, and b are parameters}
  \CommentTok{\# burnin is number of iterations to use as burnin}
\CommentTok{\# Output: theta, z}
\NormalTok{sampleGibbs \textless{}{-}}\StringTok{ }\ControlFlowTok{function}\NormalTok{(z, c, n.iter, init.theta, init.miss, r, a, b, }\DataTypeTok{burnin =} \DecValTok{1}\NormalTok{)\{}
\NormalTok{  z.sum \textless{}{-}}\StringTok{ }\KeywordTok{sum}\NormalTok{(z)}
\NormalTok{  m \textless{}{-}}\StringTok{ }\KeywordTok{length}\NormalTok{(c)}
\NormalTok{  n \textless{}{-}}\StringTok{ }\KeywordTok{length}\NormalTok{(z) }\OperatorTok{+}\StringTok{ }\NormalTok{m}
\NormalTok{  miss.vals \textless{}{-}}\StringTok{ }\NormalTok{init.miss}
\NormalTok{  res \textless{}{-}}\StringTok{ }\KeywordTok{matrix}\NormalTok{(}\OtherTok{NA}\NormalTok{, }\DataTypeTok{nrow =}\NormalTok{ n.iter, }\DataTypeTok{ncol =} \DecValTok{1} \OperatorTok{+}\StringTok{ }\NormalTok{m)}
  \ControlFlowTok{for}\NormalTok{ (i }\ControlFlowTok{in} \DecValTok{1}\OperatorTok{:}\NormalTok{n.iter)\{}
\NormalTok{    var.sum \textless{}{-}}\StringTok{ }\NormalTok{z.sum }\OperatorTok{+}\StringTok{ }\KeywordTok{sum}\NormalTok{(miss.vals)}
\NormalTok{    theta \textless{}{-}}\StringTok{ }\KeywordTok{rgamma}\NormalTok{(}\DecValTok{1}\NormalTok{, }\DataTypeTok{shape =}\NormalTok{ a }\OperatorTok{+}\StringTok{ }\NormalTok{n}\OperatorTok{*}\NormalTok{r, }\DataTypeTok{rate =}\NormalTok{ b }\OperatorTok{+}\StringTok{ }\NormalTok{var.sum)}
\NormalTok{    miss.vals \textless{}{-}}\StringTok{ }\KeywordTok{sapply}\NormalTok{(c, }\ControlFlowTok{function}\NormalTok{(x) \{}\KeywordTok{sampleTrunGamma}\NormalTok{(x, r, theta)\})}
\NormalTok{    res[i,] \textless{}{-}}\StringTok{ }\KeywordTok{c}\NormalTok{(theta, miss.vals)}
\NormalTok{  \}}
  \KeywordTok{return}\NormalTok{(res[burnin}\OperatorTok{:}\NormalTok{n.iter,])}
\NormalTok{\}}
\CommentTok{\# set parameter values}
\NormalTok{r \textless{}{-}}\StringTok{ }\DecValTok{10}
\NormalTok{a \textless{}{-}}\StringTok{ }\DecValTok{1}
\NormalTok{b \textless{}{-}}\StringTok{ }\DecValTok{1}
\CommentTok{\# input data}
\NormalTok{z \textless{}{-}}\StringTok{ }\KeywordTok{c}\NormalTok{(}\FloatTok{3.4}\NormalTok{,}\FloatTok{2.9}\NormalTok{,}\FloatTok{1.4}\NormalTok{,}\FloatTok{3.2}\NormalTok{,}\FloatTok{1.8}\NormalTok{,}\FloatTok{4.6}\NormalTok{,}\FloatTok{2.8}\NormalTok{)}
\NormalTok{c \textless{}{-}}\StringTok{ }\KeywordTok{c}\NormalTok{(}\FloatTok{1.2}\NormalTok{,}\FloatTok{1.7}\NormalTok{,}\FloatTok{2.0}\NormalTok{,}\FloatTok{1.4}\NormalTok{,}\FloatTok{0.6}\NormalTok{)}
\NormalTok{n.iter \textless{}{-}}\StringTok{ }\DecValTok{200}
\NormalTok{init.theta \textless{}{-}}\StringTok{ }\DecValTok{1}
\NormalTok{init.missing \textless{}{-}}\StringTok{ }\KeywordTok{rgamma}\NormalTok{(}\KeywordTok{length}\NormalTok{(c), }\DataTypeTok{shape =}\NormalTok{ r, }\DataTypeTok{rate =}\NormalTok{ init.theta)}
\CommentTok{\# run sampler}
\NormalTok{res \textless{}{-}}\StringTok{ }\KeywordTok{sampleGibbs}\NormalTok{(z, c, n.iter, init.theta, init.missing, r, a, b)}
\end{Highlighting}
\end{Shaded}

In figure \ref{fig:trace-theta} and \ref{fig:trace-z} we see traceplots
for 200 iterations of the Gibbs sampler. It is difficult to tell whether
or not the sampler has failed to converge, thus, we turn to running
average plots.

\begin{figure}
\centering
\includegraphics{C:/Users/nitsu/Git_Repositories/modern-bayes/homeworks/homework-6/template-hw-06_files/figure-latex/fig:trace-theta-1.pdf}
\caption{\label{fig:trace-theta}Traceplot of theta}
\end{figure}

In figures \ref{fig:run-theta} and \ref{fig:run-z} we see running
average plots for 200 iterations of the Gibbs sampler, where from all of
these it is clear that after 200 iterations the sampler is having mixing
issues, and should be run for long to check that ``it has not failed to
converge.''

\begin{figure}
\centering
\includegraphics{C:/Users/nitsu/Git_Repositories/modern-bayes/homeworks/homework-6/template-hw-06_files/figure-latex/fig:trace-z-1.pdf}
\caption{\label{fig:trace-z}Traceplot of
\(z_3, z_8, z_9, z_{10}, z_{12}.\)}
\end{figure}

\begin{Shaded}
\begin{Highlighting}[]
\CommentTok{\# get running averages}
\NormalTok{run.avg \textless{}{-}}\StringTok{ }\KeywordTok{apply}\NormalTok{(res, }\DecValTok{2}\NormalTok{, cumsum)}\OperatorTok{/}\NormalTok{(}\DecValTok{1}\OperatorTok{:}\NormalTok{n.iter)}
\end{Highlighting}
\end{Shaded}

\begin{figure}
\centering
\includegraphics{C:/Users/nitsu/Git_Repositories/modern-bayes/homeworks/homework-6/template-hw-06_files/figure-latex/fig:run-theta-1.pdf}
\caption{\label{fig:run-theta}Running average plot of theta}
\end{figure}

\begin{figure}
\centering
\includegraphics{C:/Users/nitsu/Git_Repositories/modern-bayes/homeworks/homework-6/template-hw-06_files/figure-latex/fig:run-z-1.pdf}
\caption{\label{fig:run-z}Running average plots of
\(z_3, z_8, z_9, z_{10}, z_{12}.\)}
\end{figure}

Figures \ref{fig:density-theta} and \ref{fig:density-z} do not provide
meaniful inference at this point since the sampler has not been run long
enough.

\begin{figure}
\centering
\includegraphics{C:/Users/nitsu/Git_Repositories/modern-bayes/homeworks/homework-6/template-hw-06_files/figure-latex/fig:density-theta-1.pdf}
\caption{\label{fig:density-theta}Estimated posterior density of theta}
\end{figure}

\begin{figure}
\centering
\includegraphics{C:/Users/nitsu/Git_Repositories/modern-bayes/homeworks/homework-6/template-hw-06_files/figure-latex/fig:density-z-1.pdf}
\caption{\label{fig:density-z}Estimated posterior density of \(z_9\)
(posterior mean in red).}
\end{figure}

\end{document}
